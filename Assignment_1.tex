\documentclass{article}
\usepackage{graphicx} %Required for inserting images
\usepackage{amsmath}

\title{AI110 Assignment 1}
\author{Vevek Manda }
\date{EE22BTECH11220}

\begin{document}
	
	
	
	\maketitle
	
	\section*{Question:}
	
	10.13.3.25:A coin is tossed 3 times. List the possible outcomes.\\
	Find the {probability} of getting
	(i) all heads (ii) at least 2 heads
	
	\section*{Solution:}
	The possible outcomes may be listed as follows:\\
	$\begin{pmatrix}
		{H,H,T} & {H,H,H}\\
		{H,T,H} & {H,T,T}\\
		{T,H,T} & {T,H,H}\\
		{T,T,H} & {T,T,T}
	\end{pmatrix}\\[3pt]$
	The sample space $S$ has a total of eight cases.
	
	\subsection*{(i).}
	\begin{enumerate}
	    
		\item In order to solve this questions let us take a random variable $x$ to represent the event of getting a head on a coin toss.\\[6pt]
		
		\item In this approach we will be using binomial distribution which we may state this as a fixed number of independent and identically distributed Bernoulli trials or binomial cdf which evaluates the distribution function of a binomial random variable with parameters n and p by summing probabilities of the random variable taking on the specific values in its range. These probabilities may be computed by the following recursive relationship:\\[4pt]
 $\Pr(X=j)=\frac{(n+1-j)p}{j(1-p)}\Pr(X=j-1)$\\[4pt]
 \item An table displaying the input parameters is given below:\\[7pt]
 \resizebox{\textwidth}{!}{
 \begin{tabular}{|l|l|l|}
  \hline
  Input Parameter & Value & Description\\
  \hline
  k & $\Pr(x)\geq 2$ & Argument for which the binomial distribution function is to be evaluated\\
  \hline
  n & 3 & Number of Bernoulli trials\\
  \hline
  p & $\frac{1}{2}$ & Probability of success on each trial\\[3pt]
  \hline
  \end{tabular}}\\[4pt]
		
		
		\item Using the formula for binomial probability distribution we can say that $\Pr(x)={n\choose x}(\Pr(H))^x(\Pr(T))^{n-x}$. Substituting n as 3 and x as 3 for the first part of the question we get $\Pr(0)=(\frac{1}{2})^3=\frac{1}{8}$.
	\end{enumerate}
	
	\subsection*{(ii).}
	\begin{enumerate}
	  
		\item The probability of getting at least 2 heads or $\Pr(x)\geq 2$ is equal to $\sum_{x=2}^{n}\Pr(x)$. \\[4pt]
		\item Substituting n as 3 and x as 2 and n as 3 and x as 3 for the second case we get $\Pr(2)+\Pr(3)={3\choose 2}(\frac{1}{2})^2(\frac{1}{2})^1+(\frac{1}{2})^3=\frac{1}{2}$.\\[4pt]
		\item The probability of getting at least 2 heads is therefore $\dfrac{1}{2}$.
	\end{enumerate}
\end{document}
